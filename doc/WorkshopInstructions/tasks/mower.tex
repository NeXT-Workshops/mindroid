	\section{Mähroboter}
	\ifthenelse{\boolean{tasks}}{
%	\easygcenter{.8\textwidth}{img/task_maehroboter.jpg}
	Nutze deine Kenntnisse, um den Roboter in einem mit schwarzem (oder weißem) Klebeband abgesperrten Bereich herumfahren zu lassen (so ähnlich wie beispielsweise viele Mähroboter arbeiten).
	\begin{enumerate}
	\item Wie beim Wand Ping-Pong soll der Roboter erstmal geradeaus fahren.
	\item Wenn er eine Grenze erkennt, soll er zurücksetzen und sich eine neue Richtung aussuchen.
	\item Beginne bei 1.
	\end{enumerate}
	
	Tipp: Überprüfe, welche Color-IDs auf dem Boden und den Begrenzungen erkannt werden, um festzustellen, wann der Roboter an die Umzäunung gelangt ist.
	
	Bewertungskriterien:
	\begin{itemize}
	\item (1P) Der umgrenzte Bereich darf nicht verlassen werden! Keines der Räder darf die Umgrenzung berühren!
	\item (1P) Vergleich: Wer schafft es am schnellsten, alle vier Seiten eines viereckigen Bereichs zu “treffen”?
	\end{itemize}}{}
	\ifthenelse{\boolean{solution}}{
		\sol{LawnMower}
	}{}